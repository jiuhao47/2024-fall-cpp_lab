\documentclass[../main.tex]{subfiles}
\begin{document}
\section{实验一}

\subsection{题目}

\kaishu

\begin{enumerate}
	\item 完善Safe C语言的ANTLR4词法文件(\texttt{SafeCLexer.g4}),生成token流。
	\item 完善Safe C语言的ANTLR4语法文件(\texttt{SafeCParser.g4}),生成语法分析树。
	\item 根据已有框架,修改\texttt{AstBuilder.cpp},实现对语法分析树的遍历,生成\texttt{Json}表示的抽象语法树。
\end{enumerate}

\subsection{文法}

在本实验中,我们设计并实现了一个简单编程语言的词法分析器(Lexer)和语法分析器(Parser),以探讨编译原理中词法分析和语法分析的基本概念。词法分析器负责将源代码分解成词法单元(tokens),语法分析器则根据文法规则检查这些词法单元的顺序是否符合语言的语法结构,并生成语法树(AST)。实验的目标是实现一个简单语言的基本词法分析和语法分析功能,为后续的编译优化与代码生成打下基础。

\subsection{词法分析器(Lexer)}

词法分析器的主要任务是将源代码分解成一系列词法单元。在本实验中,词法单元包括标识符、常量、运算符、分隔符等,具体的词法规则如下:

\begin{itemize}
	\item 标识符(Identifier):由字母或下划线开头,后面可以跟字母、数字或下划线,表示变量名、函数名等。
	      \[
		      \text{Identifier} \colon [\_a-zA-Z][a-zA-Z0-9\_]^*
	      \]

	\item 整数常量(IntConst):可以是十进制或十六进制表示的数字。
	      \[
		      \text{IntConst} \colon (0x | 0X)[0-9a-fA-F]+ \mid [0-9]+
	      \]

	\item 关键字和符号:如逗号(\texttt{,})、分号(\texttt{;})、赋值符号(\texttt{=})、括号、运算符等。
\end{itemize}

词法分析器的任务是扫描源代码并将其分解成这些基本单元,生成一系列的词法单元序列供语法分析器进一步处理。

\subsection{语法分析器(Parser)}

语法分析器的任务是根据给定的文法规则检查词法单元序列是否符合语言的语法结构,并构建语法树(AST)。在本实验中,我们根据词法分析器生成的词法单元,设计了一个符合特定语法规则的语法分析器。以下是语法规则的详细描述。

\subsubsection{编译单元(compUnit)}

首先,整个程序的语法结构由多个声明(\texttt{decl})和函数定义(\texttt{funcDef})组成,程序以文件结束符(EOF)结尾。该语法规则可以表示为:

\[
	\text{compUnit} \colon (\text{decl} \mid \text{funcDef})^+ \ \text{EOF}
\]

其中,\texttt{decl} 表示声明,\texttt{funcDef} 表示函数定义。

\subsubsection{函数定义(funcDef)}

函数定义的语法规则定义了函数的结构,包括返回类型、函数名、参数列表以及函数体。根据语法规则,函数定义的语法为:

\[
	\texttt{funcDef} \colon \texttt{Void} \ \texttt{Identifier} \ \texttt{LeftParen} \ \texttt{RightParen} \ \texttt{block}
\]

其中:
\begin{enumerate}
	\item \texttt{Void} 表示函数返回类型为 \texttt{void};
	\item \texttt{Identifier} 表示函数的标识符(函数名);
	\item \texttt{LeftParen} 和 \texttt{RightParen} 表示参数列表的左括号和右括号。此处的语法规则表明函数没有参数(即参数列表为空);
	\item \texttt{block} 表示函数的代码块,即函数体,其中包含了函数的执行语句。
\end{enumerate}

因此,函数定义语法规则指定了一个没有参数的 \texttt{void} 类型函数,它的代码块包括该函数的实际操作。


\subsubsection{声明 decl}

声明分为常量声明(\texttt{constDecl})和变量声明(\texttt{varDecl})。其语法规则为:

\[
	\text{decl} \colon \text{constDecl} \mid \text{varDecl}
\]

常量声明的语法规则为:

\[
	\text{constDecl} \colon \text{Const} \ \text{bType} \ \text{constDef} \ (\text{Comma} \ \text{constDef})^* \ \text{SemiColon}
\]

其中,\texttt{bType} 表示基本类型(如整数类型),\texttt{constDef} 表示常量定义。语法规则如下:

\[
	\text{constDef} \colon (\text{Identifier} \mid \text{array}) \ \text{Assign} \ (\text{exp} \mid (\text{LeftBrace} \ \text{exp} (\text{Comma} \ \text{exp})^* \ \text{RightBrace}))
\]

常量可以是标识符或数组,且必须有一个初始值(可以是单个值或数组)。

变量声明的语法规则为:

\[
	\text{varDecl} \colon \text{bType} \ \text{varDef} \ (\text{Comma} \ \text{varDef})^* \ \text{SemiColon}
\]

变量定义的语法规则为:

\[
	\text{varDef} \colon (\text{Identifier} \mid \text{array}) \ (\text{Assign} \ (\text{exp} \mid (\text{LeftBrace} \ \text{exp} (\text{Comma} \ \text{exp})^* \ \text{RightBrace})) )?
\]

其中,变量可以是标识符或数组,并且可以有初始值。

\subsubsection{数组声明(array)}

数组声明有两种形式:obc 数组和普通数组。obc 数组的语法规则为:

\[
	\text{obcArray} \colon \text{obc} \ \text{unobcArray}
\]

普通数组的语法规则为:

\[
	\text{unobcArray} \colon \text{Identifier} \ \text{LeftBracket} \ (\text{exp})? \ \text{RightBracket}
\]

其中,\texttt{obcArray} 是 \texttt{obc} 类型的数组,\texttt{unobcArray} 是普通数组,后者可能包括数组的大小。

\subsubsection{语句(stmt)}

语句的语法规则如下,涵盖了多种语句类型,包括赋值语句、条件语句、循环语句等:

\begin{gather*}
	\begin{aligned}
		\text{stmt} & \colon  \text{block}                                                                                             \\
		            & \mid \text{lval} \ \text{Assign} \ \text{exp} \ \text{SemiColon}                                                 \\
		            & \mid \text{SemiColon}                                                                                            \\
		            & \mid \text{exp} \ \text{SemiColon}                                                                               \\
		            & \mid \text{If} \ \text{LeftParen} \ \text{cond} \ \text{RightParen} \ \text{stmt} \ (\text{Else} \ \text{stmt})? \\
		            & \mid \text{While} \ \text{LeftParen} \ \text{cond} \ \text{RightParen} \ \text{stmt}                             \\
		            & \mid \text{Identifier} \ \text{LeftParen} \ \text{RightParen} \ \text{SemiColon}
	\end{aligned}
\end{gather*}

其中,\texttt{block} 是语句块,\texttt{lval} 是左值,\texttt{exp} 是表达式,\texttt{cond} 是条件表达式,\texttt{If} 和 \texttt{While} 分别表示条件语句和循环语句。\texttt{Else} 部分是条件语句的可选部分。\texttt{Identifier} \texttt{LeftParen} \texttt{RightParen} \texttt{SemiColon} 表示函数调用。

\subsubsection{块(block)}

在编程语言中,语句块是由一对大括号包围的一组语句。语句块是程序中的基本组成部分,用于组织一组语句,使它们可以作为一个整体执行。语法规则如下:

\[
	\texttt{block} \colon \texttt{LeftBrace} \ (\texttt{blockItem})^* \ \texttt{RightBrace}
\]

其中:
\begin{enumerate}
	\item \texttt{LeftBrace} 和 \texttt{RightBrace} 分别表示语句块的开始和结束;
	\item \texttt{blockItem} 表示语句块内的单个语句或声明,可以是声明(\texttt{decl})或语句(\texttt{stmt})。
\end{enumerate}

这意味着一个语句块可以包含零个或多个语句或声明。通过这种方式,语句块允许程序结构化地组织代码。

\subsubsection{左值(lval)}

左值是可以出现在赋值语句中的表达式,它表示一个可以被赋值的变量。左值通常是一个标识符(变量名)或者是数组元素、函数调用等。左值的语法规则如下:

\begin{gather*}
	\begin{aligned}
		\text{lval} & \colon \text{Identifier}                                                         \\
		            & \mid (\text{Identifier} \ \text{LeftBracket} \ \text{exp} \ \text{RightBracket})
	\end{aligned}
\end{gather*}

其中:

\begin{enumerate}
	\item \texttt{Identifier} 表示一个简单的标识符,即一个变量名;
	\item \texttt{(Identifier LeftBracket exp RightBracket)} 表示数组元素的访问,\texttt{exp} 表示数组索引的表达式;
\end{enumerate}

例如,对于表达式 \texttt{a = 10;} 中,\texttt{a} 是左值;在表达式 \texttt{a[5] = 10;} 中,\texttt{a[5]} 也是一个左值,因为它表示数组 \texttt{a} 中索引为 \texttt{5} 的元素。


\subsubsection{表达式(exp)}

表达式的语法规则非常重要,它涵盖了变量、常量、运算符等基本构件。表达式的语法规则如下:

\begin{gather*}
	\begin{aligned}
		\text{exp} & \colon  \text{lval}                                                                                                                                    \\
		           & \mid \text{number}                                                                                                                                     \\
		           & \mid \text{LeftParen} \ \text{exp} \ \text{RightParen}                                                                                                 \\
		           & \mid (\text{Minus} \mid \text{Plus}) \ \text{exp}                                                                                                      \\
		           & \mid \text{exp} \ (\text{Multiply} \mid \text{Divide} \mid \text{Modulo}) \ \text{exp}                                                                 \\
		           & \mid \text{exp} \ (\text{Plus} \mid \text{Minus}) \ \text{exp}                                                                                         \\
		           & \mid \text{exp} \ (\text{Equal} \mid \text{NonEqual} \mid \text{Less} \mid \text{Greater} \mid \text{LessEqual} \mid \text{GreaterEqual}) \ \text{exp}
	\end{aligned}
\end{gather*}

其中,\texttt{lval} 表示左值,\texttt{number} 表示常量,\texttt{Plus}、\texttt{Minus}、\texttt{Multiply}、\texttt{Divide}、\texttt{Modulo} 表示运算符,\texttt{Equal}、\texttt{NonEqual}、\texttt{Less}、\texttt{Greater} 等表示关系运算符\footnote{笔者将\texttt{cond}的 实现纳入了\texttt{exp}}。


\songti
\subsection{工程修改说明}

本次实验笔者添加了\texttt{Utils.h}用于定义调试输出宏。

\begin{mdframed}
	\begin{minted}[breaklines,breakanywhere]{cpp}
#define JJY_DEBUG_AST 0
#define JJY_DEBUG_IR 1
#define JJY_DEBUG_OBC_CHECK 1
#define JJY_DEBUG_SIGN "[j] "
  \end{minted}
\end{mdframed}

本次实验未修改除了\texttt{SafeCLexer.g4}、\texttt{SafeCParser.g4}、\texttt{AstBuilder.cpp}之外的任何文件。

\subsection{实现过程及困难}

\subsubsection{SafeCLexer.g4}

本文件主要是对Safe C语言的词法分析器进行定义,包括了Safe C语言的关键字、标识符、常量、运算符等的词法规则。

本文件的编写过程中并没有遇到太大的困难,主要是根据Safe C语言的词法规则进行定义。

\subsubsection{SafeCParser.g4}

本文件的编写过程主要在于理解Safe C语言的语法规则,并将其转化为ANTLR4的语法规则。(呈现为token流的方式)

本文件的完善过程主要是循序渐进的根据测试脚本的执行结果进行逐步的完善。测试脚本会给出当前语法分析器在解析文件时的错误信息(通常为遇到的未定义token的位置),根据这些信息,找到对应的规则进行Debug。

本文件的编写过程中遇到的主要困难是对Safe C语言的语法规则的理解,以及如何将其转化为ANTLR4的语法规则。在编写过程中,我们发现了一些Safe C语言的语法规则与C语言的语法规则不同的地方,例如Safe C语言中的数组声明需要在变量名后面加上\texttt{obc}关键字,这在C语言中是不需要的。而且在编写\texttt{AstBuilder.cpp}时,我们发现之前通过测试的Safe C语法规则在实现抽象语法树构建时会引入不必要的麻烦与冗余(主要体现在对于赋值语句被错误的划分到表达式逻辑中),故我们在编写\texttt{AstBuilder.cpp}过程中对语法规则进行了修改。

\begin{mdframed}
	\begin{minted}[breaklines,breakanywhere]{antlr}
stmt: 
    block
    | lval Assign exp SemiColon // 此行于编写AstBuilder.cpp时添加
    | SemiColon
    | exp SemiColon
    | If LeftParen cond RightParen stmt (Else stmt)?
    | While LeftParen cond RightParen stmt;
exp:
    lval
    | lval Assign exp // 此行于编写AstBuilder.cpp时删去
    | number
    | LeftParen exp RightParen
    | exp (Plus | Minus | Multiply | Divide | Modulo) exp
    | (Minus | Plus) exp
    | exp (Equal | NonEqual | Less | Greater | LessEqual | GreaterEqual) exp;
  \end{minted}
\end{mdframed}

此外在进行Lab2,中间代码生成实验时,我们发现上述文法依旧存在问题

对于\texttt{stmt}的文法规则,我们将\texttt{lval}规则中的函数调用部分划分到了\texttt{stmt}中,这样在实现中间代码生成时,我们可以直接将函数调用作为一个\texttt{stmt}进行处理。

\begin{mdframed}
	\begin{minted}[breaklines,breakanywhere]{antlr}
lval: 
    Identifier 
    | (Identifier LeftBracket exp RightBracket)
    | (Identifier LeftParen RightParen); // 此行于编写SafeCIRBuilder.cpp时删去

stmt: 
    block
    | lval Assign exp SemiColon
    | SemiColon
    | exp SemiColon
    | If LeftParen cond RightParen stmt (Else stmt)?
    | While LeftParen cond RightParen stmt
    | Identifier LeftParen RightParen SemiColon; // 此行于编写SafeCIRBuilder.cpp时添加
  \end{minted}
\end{mdframed}

对于\texttt{exp}的文法规则,我们发现其优先级存在问题,我们将其进行了修改。

\begin{mdframed}
	\begin{minted}[breaklines,breakanywhere]{antlr}
exp:
    lval
    | number
    | LeftParen exp RightParen
    | (Minus | Plus) exp // 此行于编写SafeCIRBuilder.cpp时添加,用于优先处理正负号
    | exp (Multiply | Divide | Modulo) exp // 此行于编写SafeCIRBuilder.cpp时添加,用于优先处理乘、除、模运算
    | exp (Plus | Minus) exp
    | exp (Equal | NonEqual | Less | Greater | LessEqual | GreaterEqual) exp;
  \end{minted}
\end{mdframed}



\subsubsection{AstBuilder.cpp}

本文件的编写过程主要是根据ANTLR4生成的语法分析树,遍历语法分析树,生成抽象语法树。

本文件的编写过程先借助\texttt{Copilot}对整体代码结构进行了建构,但是这一版代码并不能正确执行功能,其会遇到的主要问题有:

\begin{enumerate}
	\item 不能进入正确的建构逻辑片段,导致节点类型与内容错误
	\item 不能进行正确的类型转换,出现\texttt{bad cast}错误
	\item 不能正确的实现抽象语法树节点的字段填写,出现空字段,从而在建构逻辑正确的情况下,输出抽象语法树时出现\texttt{Segmentation Fault}错误
\end{enumerate}

\textbf{首先我们简述我们调试分析过程中插入的终端输出代码:}

\begin{mdframed}
	\begin{minted}[breaklines,breakanywhere]{cpp}
if (JJY_DEBUG_AST)
  printf("%s %s %d %d\n", JJY_DEBUG_SIGN, __func__, result->line, result->pos);
  \end{minted}
\end{mdframed}


\textbf{下面简述我们调试分析建构逻辑错误的过程:}

\begin{enumerate}
	\item 编译运行,观察终端输出,初步了解本次运行的函数调用情况
	\item 通过手动跟踪调用过程,发现错误的分支进入,定位错误发生的行号
	\item 修改,回到步骤1
\end{enumerate}


\textbf{下面简述我们调试分析类型转换错误的过程:}

\begin{enumerate}
	\item 编译运行,观察预定义的终端输出,初步了解本次运行的函数调用情况
	\item 使用gdb调试运行,使用\texttt{bt}指令查看函数调用栈,定位错误发生的行号
	\item 通过手动跟踪调用过程,观察类型转化流程,发现错误的类型转化
	\item 修改,回到步骤1
\end{enumerate}

\begin{mdframed}
	\begin{minted}[breaklines,breakanywhere]{cpp}
antlrcpp::Any AstBuilder::visitNumber(SafeCParser::NumberContext *ctx) {
  auto result = new number_node;
  /* 略去其它代码 */
  // return result;
  // 需要进行正确的类型转换(子类到父类的转换)
  return dynamic_cast<expr_node *>(result);
}
  \end{minted}
\end{mdframed}


\textbf{下面简述我们调试分析空字段错误的过程:}

\begin{enumerate}
	\item 编译运行,观察预定义的终端输出,发现\texttt{Segmentation Fault}错误
	\item 使用gdb调试运行,发现是在写入抽象语法树\texttt{Json}时出现错误
	\item 此时只能手动排查每一个涉及的抽象语法树节点,查找可能的空字段\footnote{起初我们并未发现是空字段问题,是在逐个排查抽象语法树节点时偶然发现}
	\item 修改,回到步骤1
\end{enumerate}

\begin{mdframed}
	\begin{minted}[breaklines,breakanywhere]{cpp}
antlrcpp::Any AstBuilder::visitConstDef(SafeCParser::ConstDefContext *ctx){
  /* 前文省略 */
  result->btype = BType::INT;
  // 此字段若不进行赋值,会导致后续在输出抽象语法树时出现Segmentation Fault错误
  /* 后文省略 */
}
  \end{minted}
\end{mdframed}


\end{document}
